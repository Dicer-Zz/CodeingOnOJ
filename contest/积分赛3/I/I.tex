\documentclass[12pt, a4paper]{article}
\usepackage[UTF8, scheme = plain]{ctex}
\renewcommand\thesection{\Alph{section}} \renewcommand\thesubsection{\thesection.\arabic{subsection}}
\usepackage{geometry}
\geometry{left=1.5cm,right=2cm,top=2cm,bottom=1.5cm}

% Title
\title{暑期集训第一次积分训练赛}
\author{河南理工大学ACM协会 \thanks{name1 name2}}
\date{2018/7/13}

\begin{document}

\maketitle\newpage

% *************************************      仅修改以下内容        *******************************************

\section{阶乘之和}

\begin{table}[!h]
  \centering
  \begin{tabular}{l|l|l}
  时间限制 & 内存限制 & 出题人 \\
  \hline
  1 Second & 512 Mb & 凡凯 \\
\end{tabular}
\end{table}

\subsection*{题目描述}

% \newline %换行命令
对于整数$p$,给出以下定义\\
$p=x_{1}!+x_{2}!+x_{3}!+...+x_{q}!(x_{i}<x_{j}for\ all\ i<j )$且$x_{i} \neq 0$\\
$($注释:p等于多个数的阶乘和,并且$x_{1},x_{2},x_{3},...,x_{q}$为不相等的任意正整数,即组成p的阶乘不重复使用$)$\\
给定两个整数x,y,判断二者是否能满足以上定义。若二者都满足定义,设x由$k_{1}$个数的阶乘和组成,y由$k_{2}$个数的阶乘和组成,若$k_{1}=k_{2}$,按下述输出格式输出二者的定义形式$($输出时,阶乘按递增形式输出,例如:7=1!+3!$)$。

\subsection*{输入}

第一行输入一个整数T,代表T组测试数据。$(1\leq T \leq 10000)$\\
接下来T行,每行包含两个整数x,y。$(1\leq x,y \leq 10^{18})$\\

\subsection*{输出}
对于每组x,y输出包含两部分:\\
①如果二者都满足以上定义,输出“SEYES”;如果只有其一满足以上定义,输出“YNEOS”;如果二者都不满足以上定义,则输出“ONO”。\\
②当x,y都满足以上定义且$k_{1}=k_{2}$时,输出二者的定义形式。否者输出“dWvWb”。


\subsection*{输入样例}
4\\
7 7\\
1 2\\
4 2\\
4 4\\


\subsection*{输出样例}
Case 1:SEYES\\
7=1!+3!\\
7=1!+3!\\
Case 2:SEYES\\
1=1!\\
2=2!\\
Case 3:YNEOS\\
dWvWb\\
Case 4:ONO\\
dWvWb\\

% 有提示内容的话就去掉下一行行首的注释:符百分号
% \subsection*{提示}

\end{document} 


% latex数学符号表 https://blog.csdn.net/u013346007/article/details/54138690